\section{L'interface graphique}
Nous somme désolé de vous apprendre que l'interface graphique n'est pas ce qu'elle aurait du être, elle est fonctionnelle, certe, mais pas aussi fonctionnelle que nous l'aurions voulu et aussi avancée que nous l'imaginions.\\
Mais commençons par le commencement, qu'est ce qu'une interface graphique ?

	\subsection{Interface graphique}
	Une interface graphique ou GUI (graphical user interface) en anglais, est un dispositif permetant une meilleur compréhension des programmes utilisé grâce a la manipulation d'objet dessiné a l'écran. Elle a été créé par les ingénieurs du Xerox PARC à la fin des années 1970, pour remplacer les interfaces en ligne de commande, puis développé et popularisé par Apple avec l'ordinateur Macintosh, commercialisé en 1984.

	\subsection{Langage utilisé}
		Pour le projet CamelT'OCT, nous avons décidé d'utiliser Caml associé à la librairie GTK pour créer l'interface grafique. L'utilisation de GTK n'est pas une mince affaire, car il faut bien commprendre comment il fonctionne, ce qui est possible, les limites ainsi que l'ordre du code.

	\subsection{Le commencement}
	Les débuts avec GTK ont été assez difficile. En effet, la documentation n'étant pas aussi abondante que dans les autres domaines, il à été assez difficile de comprendre comment fonctionnait GTK. Même à l'aide d'exemple guidé, l'apprentisage de la synthaxe ainsi que la compréhension des diférente fonction créée sont apparu comme des obstacle majeur pour faire l'interface.\\
	Il faut comprendre le principe de GTK pour avancer avec GTK, mais il faut avancer pour comprendre, ce qui rend la tache assez difficile quand on ne saisi pas la notion de Widget. Un Widget (Window Gadget) est une structure deffinissant les propriétés d'un objet associé a une panoplie de fonctions permettant de manipuler cet objet. Les Widget marche un peu comme des boites, d'ailleur, on peut voir une interface graphique comme des boites dans des boites (des poupés russes en plus complex vu que certaine poupé peuvent contenir plus d'une poupé qui en contiennent chaqu'une un nombre propre). De plus, chaque boite à ces proporiété, ce qui rends la compréhension encore plus flou. \\
	Malgré toutes ces confusions, nous avons quand même réussi à réalisé une interface graphique qui tiens la route, même si ce n'est pas le résultat que nous attendions.

	\subsection{La réalisation}
	Pour réaliser notre interface graphique, nous avons utilisé les exemples de plusieur site internet, mais le site qui nous à le plus aidé est à cette adresse :\\
	http://blog.developpez.com/ocamlblog/\\
	Nous allons vous montrer comment notre interface a été développé et les différent problèmes auquels nous nous somme heurté.\\
	\\
	Pour commencer, il faut créer une boite principal contenant les différente boite a venir, en gros, nous allons créer la fenêtre principale de notre application qui se fermera quand nous cliquerons sur la croix :\\

	\begin{lstlisting}
			let window =
				let win = GWindow.window
					~title: "CamlT'OCR"
					~height:700
					~width:1200 ()
					~resizable:true in
				win\#connect\#destroy GMain.quit;
				win 
	\end{lstlisting}

	Une fois cette fenêtre créer, il faut lui ajouter des éléments, hors, elle ne peut en accepter qu'un. Généralement, on lui donne un élément qui peut, lui, accépter plusieurs éléments. Nous avont choisi un boite qui range les éléments verticalement plutôt qu'horizontalement car cela nous paraissait plus logique que les boutons soit en haut et non a gauche :\\
	
	\begin{center}
	\[ let vbox = GPack.vbox\\
			~spacing:10\\
			~packing:window#add () \\\]
	\end{center}

	Le premier élément ajouté à été une bare d'outil pour stocker les boutons cliquable. Elle les disposera de façon horizontal : 

	\begin{center}
	\[ let toolbar = GButton.toolbar  \\
		  ~orientation:`HORIZONTAL\\
		  ~style:`ICONS\\
		  ~packing:vbox#pack ()\\\]
	\end{center}

	Les diférent boutons sont stockés soit dans des item qui eux même sont stocké dans la bare d'outil, soit directement dans la barre d'outil. Ce choix des items peut paraitre absurde, pourquoi s'embéter à rajouter du code si au final on peut le faire directement ? Et bien c'est parce qu'en fonction du contenu du bouton, il ne pourra pas être stocké n'importe où. par exemple :\\ 

	 \begin{center}
	\[ let button = GFile.chooser_button\\
		  ~title:"Ouvrir"\\
		  ~action:`OPEN\\
		  ~packing:item#add ()\\\]
	\end{center}

	Ce bouton ne pourra pas être mis dans la barre d'outil directement a cause de sa définition :
	\begin{center}
	\[ let button = GFile.chooser_button\\\]
	\end{center}

	alors qu'un bouton normal aura l'attribut :
	\begin{center}
	\[ let button = GButton.button\\\]
	\end{center}

	Voici l'un des nombreux problèmes auxquels nous nous somme confronté, pour trouver comment contourner le problème, nous avons dù chercher longtemps et essayer beaucoup de chose avant de trouver celle qui fonctionne le mieux.\\
	Nous allons passer a la deuxième partie de l'interface, l'affichage de l'image traité et la zone de texte, car qui dit OCR, dit forcément image.\\
	Nous avions donc un bouton pour ouvrir une image, mais aucun endroit où l'afficher. La fonction d'affichage n'étant pas très compliqué, il nous a falu peut de temps pour la coder :\\

	\begin{center}
	\[ let image = GMisc.image \\
	  ~width:560 ~height:400\\
	  ~packing:(vbox#pack) ()\\\]
	\end{center}

	Cependant, l'image était bien affiché mais elle prenait la place qu'elle voulait, et elle agrandissait la fenêtre quand elle était beaucoup trop grande. Nous avons donc pensé à la redimentionner, et fixer un taille maximal mais nous n'avons pas trouvé de fonction pour nous aider à comprendre comment faire. C'est la que nous est venu une idée, au lieu de la redimentionner, nous allions la mettre dans une boite qui permettait le défillement. Fort heureusement, une tel chose existait et cela résolu nos problème de taille (jeu de mot) aux quels nous étions confronté. Nous avons utilisé les barres de défillement également pour le texte, comme cela nous pouvons écrire autant que nous le voulons.

	\begin{center}
	\[ let scroll = GBin.scrolled_window\\
    	~hpolicy:`ALWAYS\\
    	~vpolicy:`ALWAYS\\
    	~shadow_type:`ETCHED_IN\\
    	~packing:hbox#add ()\\\]
	\end{center}

	Pour finir, ce début d'interface était assez prometteur, un chargement d'image, une zone de texte pour écrire et éditer, un bouton près a acceuilir une fonction de traitement pour l'image, et même un bouton sauvegarde qui prend le texte écrit et l'enregistre dans un fichier. Cependant, uelque problème nous ont obligé a ne pas faire ce que nous attendions.


	\subsection{La non réussite de certaine partie}

	Malheureusement, le résultat de l'interface graphique n'est pas celui que nous imaginions. En effet, dans l'idéal, nous voulions une interface qui :\\
	\begin{itemize}
		\item ouvre une image et l'affiche
		\item des boite pouvant contenir une image et du texte
		\item un bouton de traitement de l'image qui, comme son nom l'indique, applique les différente opération à l'image et affiche le texte dans un éditeur de texte
		\item un bouton de sauvegarde qui enregistre le texte créé dans un fichier.
	\end{itemize}
	Au final, nous avons les élément principaux de l'interface, à savoir l'ouverture d'une image dans une boite contenant des barres de défilements, une zone de texte où l'on peut taper ce que l'on veut, un bouton de sauvegarde du texte, qui enregistre tout le contenu de l'éditeur de texte dans un fichier à l'endroit de notre choix. Cependant, vous constaterez qu'il manque un élément principale, le bouton apliquant les fonction de traitement à l'image. Evidament, nous n'avons pas laissé un bouton vide, nous avons essayer de faire au mieux.

	\subsection{le contournement du problème tant bien que mal}
	Le bouton appele bien une fonction qui va traiter l'image,
