\documentclass{article}
\usepackage[french]{babel}
\usepackage[T1]{fontenc}
\usepackage[utf8]{inputenc}
\usepackage{graphicx}
\usepackage{eurosym}
\usepackage {times}
\usepackage{fancyhdr}
\usepackage{bm}
\usepackage{listing}
 
\setlst{language=Caml}
\title{Projet CamlT'OCR par SMK \~Cahier des charges}
\pagestyle{fancyplain} \lhead{\textit{Projet DotNet}} \rhead{\textit{KMnO4}}
\date{24-10-2014}
\author{
    Timothe \textit{Tim-Tim} Bureau-Godart(bureau\_t) \and
        Leopold \textit{Meta} Szabatura (szabat\_l) \and
        Louis \textit{Zab} Forget (forget\_l) \and
        Maxime \textit{Kylox} Gaudron (gaudro\_m)
        1      }


\begin{document}
\maketitle
\tableofcontents
\section{Introduction}
\addcontentsline{toc}{section}{Introduction}
Dans le cadre de notre specialisation en informatique a l'EPITA, nous avons eu comme projet smestrielle la realisation d'un ocr ( Optical Character Recognition). Ce projet ce deroule sur 4 mois en Ocaml, un langage develloper par l'INRIA. Nous presentons donc dans ce rapport l'etat d'avancement du projet a la vielle de la premiere soutenance. Ce projet a ete realise par quattre membre fidele et devoue a la bonne cause celle de notre ocr ! 
\subsection{presentation des membres}
\subsubsection{Louis "\textit{Zab}" Forget}
\subsubsection{Timothe "\textit{Tim-Tim}" Bureau Godart}
\subsubsection{Leopold "\textit{Meta}" Szabatura}
\subsubsection{Maxime "\textit{kylox}" Gaudron}
\subsection{organisation du projet}
\section{Les taches}
\subsection{quelque mots sur le site internet}
\subsection{Interface graphique}
\subsection{Traitement d'image}
\subsubsection{Le niveau de gris}
\subsubsubsection{\textbf{Le concept}}:\\
Une image se compose d'element appeler pixel et definie par trois composante (en realite quattres mais ceci est une autre histoire) qui sont R,G et B correspondant au valeur Red, Green et Blue d'un pixel. Cette etape est primordiale car elle va permettre le bon traitement de l'image par l'ensemble des etapes aui la succede.\\
\subsubsubsection{\textbf{La realisation}}:\\
Pour realiser ce niveau de gris on va travailler sur le trois composante d'un pixel et appliquer la fromule suivante :
\\
\begin{center}
\[x = \frac{0.299 \times R + 0.587 \times G + 0.114 \times B}{3}\]

a l'ensemble des pixels de l'image.

\end{center}
\subsubsection{Le filtre median}
\subsubsection{La binarisation}
\subsubsection{rotation}
\subsection{Decoupage de l'image}
\subsection{Perceptron}
\section{Conclusion}


\end{document}
